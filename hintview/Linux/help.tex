%\font\sans=lmsans10-regular.otf
\font\sans=cmss10
\sans
\parskip 6pt plus 2pt minus 3pt
\parindent 0pt

\newcount\labelno \labelno=1

\def\outline#1#2{\HINToutline goto num \the\labelno depth #2 {#1}%
  \HINTdest num \the\labelno\relax
  \global\advance\labelno by 1\relax}

\def\link#1{\HINTstartlink goto name {#1} #1\HINTendlink} 
\def\label#1{\HINTdest name {#1}}

\def\section#1{\bigskip\goodbreak\noindent
  \outline{#1}{0}#1\par}

\def\itemize{\par\bigskip\bgroup\leftskip 1in}
\def\item#1{\par\leavevmode
  \outline{#1}{1}\label{#1}\hbox to 0pt{\hss\hbox to 1in {#1\hfill}}\ignorespaces}
\def\enditemize{\par\egroup}



\section{NAME}
        hintview - displaying HINT files

\section{SYNOPSIS}
        hintview [options] [file]

\section{DESCRIPTION}

       Hintview displays a binary HINT file, usually with the .hnt extension.

       The  binary HINT file format is designed for on-screen reading of documents.
       Using hintview to display a HINT file, its content will dynamically  adapt to
       the available display area. For complete information on
       the  HINT  file  format  and  programs  to   view   HINT   files,   see
       https://hint.userweb.mwn.de.

\section{OPTIONS}
       This  version  of  hintview  understands the following command line options:
       
\itemize
\item{-a}     Start in autoreload mode. If in autoreload mode, the viewer will
              check the modification time of the displayed HINT file each time
              the cursor enters the window.   If  the  modification  time  has
              changed,  the file is reloaded.  The viewer attempts to position
              the new file at about the same position as the  old  file.  This
              attempt  might fail if large scale changes have been made to the
              file.

              See also the \link{Ctrl-A} and the \link{Ctrl-R} key.

\item{-h}     Open the document on the "home" page. The home page is  a  location
              specified  by the document author as a good starting point
              for reading the document.

              See also the \link{Ctrl-H} key.

\item{--help} Print help message and exit.

\item{-d}     Start the viewer in "dark" mode. In this mode the  viewer  will
              display  the  document  using  a dark background and light foreground
              color to reduce eye strain  in  a  surrounding  with  dim
              lighting.

              See also the \link{Ctrl-D} key.

\item{--version}
              Print version information and exit.

\item{-z}     Zoom  to  100\%.  The  HINT viewer can display a document using a
              given zoom factor to make characters appear larger or smaller to
              either  improve  readability or increase the amount of text that
              can be displayed in a given window. This option resets the  zoom
              factor  so  that  a  10pt font will appear on screen at the same
              size as it would appear in a printed book.

              See also the \link{Ctrl-Z} key.
              
\enditemize

\section{KEYSTROKES}
       Hintview recognizes the following keystrokes when typed in its window.

\itemize
\item{Ctrl-A} Toggle autoreload mode. See also the \link{-a} option  and  the  \link{Ctrl-R}
              key.

\item{Ctrl-D} Toggle between dark and light mode. See also the \link{-d} option.

\item{Ctrl-F} Open the Find dialog.

\item{Ctrl-G} Find the next occurence.

\item{Ctrl-H}\item{Home}
              Move to the "home" page. The home page is a  location  specified
              by  the document author as a good starting point for reading the
              document.
  
\item{Ctrl-O} Open the File dialog.

\item{Ctrl-P} Open the Preferences dialog.
  
\item{Ctrl-Q} Quit. Terminate the program.

\item{Ctrl-R} Reload the content. The file is reloaded and the viewer attempts
              to position the new file at about the same position as  the  old
              file.  This  attempt might fail if large scale changes have been
              made to the file. See also the \link{Ctrl-A} key and the \link{-a} option.

\item{Ctrl-T} Show a clickable table of content.

\item{Ctrl-Z} Zoom to 100\%. See also the \link{-z} option.


\item{Page-Down}
              Move to the next page.

\item{Page-Up}
              Move to the previous page.

\item{NUM +}  Increase the gamma correction by 0.1

\item{NUM -}  Decrease the gamma correction by 0.1
  
\enditemize


\section{MOUSE ACTIONS}
       The following actions are executed using the left mouse button.

\itemize
         
\item{Click}  If the mouse is positioned on a link, move to the  link  target.
              Links are displayed in blue color.

\item{Drag}   Dragging  from  the  top-left  to the bottom-right increases the
              zoom factor.  Dragging in the opposite direction  decreases  the
              zoom factor. See also the \link{Ctrl-Z} key and the \link{-z} option.

\item{Scroll} wheel (EXPERIMENTAL)
              Using the scroll wheel is a fast way of paging forward and backward.
              
\enditemize

              
\section{NOTES}
       This is a program to render HINT files using GTK-3. Other  programs  to
       display  HINT  files  are  available  on  the HINT project home page at
       https://hint.userweb.mwn.de/ where you find viewers for the Windows and
       Android operating systems.

       Currently  the  best  way to produce HINT files is the use of hitex, or
       hilatex, versions of TeX that produce HINT files as  output.  The  HINT
       file format is described in HINT: The file format which is available as
       a book or in electronic form from the HINT project home page. The  HINT
       file format is independent of TeX and allows the implementation of
       generators for all kind of word-processors. The hintview  program  however
       uses parts of Donald E. Knuth's implementation of TeX, notably its line
       breaking routine, to produce good quality rendering of documents.

\section{AVAILABILITY}
       hintview is tested on the Linux platform but should compile on a  large
       variety  of machine architectures and operating systems.  hitex and hilatex
       are part of the TeX Live distribution.

%\section{SEE ALSO}
%       hitex(1), histretch(1), hishrink(1).

\section{AUTHORS}
       Martin Ruckert

\vfill       
\end
